\documentclass[a4paper,11pt]{scrartcl}

\usepackage[utf8]{inputenc}
\usepackage[T1]{fontenc}
\usepackage{amsmath}
\usepackage[left=3cm, right=2cm, top= 2cm, bottom = 2.5cm, includeheadfoot]{geometry}
\usepackage[svgnames]{xcolor}

\usepackage{pdfpages}
\usepackage{listings} 
\lstset{
	basicstyle=\scriptsize\color{Black},
	keywordstyle=\color{SteelBlue}\bfseries,
	language = java,
	numbers=right, numberstyle=\small, stepnumber=5, numbersep=0pt,
	showstringspaces=false}
\usepackage{url}
\usepackage{setspace}
%umgebungen sind \begin(singlespace/onehalfspace/doublespace)
%einfacher eineinhalbfacher und 2facher zeilenabstand
%es kann aber auch nur mit \singlespace \doublespace und \onehalfspacing gearbeitet werden.

\usepackage{times}
\usepackage{graphicx}
\usepackage{color}
\usepackage{fancyhdr}
\pagestyle{fancy}
\lhead{\small Jonas Kordt, Simon Stadlinger}
\chead{\small Serie 5}
\rhead{\small Programmiertechnik II}

\rfoot{}
\cfoot{\thepage}
\lfoot{}

%\pagestyle{headings}

%Befehle um Kopfund Fuszeile Anzupassen:
%\thepage - Seitennummer
%\leftmark - Aktueller Kapitelname "Kapitel X. Name des Kapitels"
%\rightmark - Aktueller Untertitelname "X.X Name des Kapitels
%\chaptername - Das Wort "Kapitel"
%\thechapter - Aktuelle Kapitelnummerierung
%\thesection - Aktuelle Unterkapitelnummerierung
%\slshape - Kursiv und Sanseriv


\begin{document}
\begin{center}
{\LARGE Zusatzaufgabe Code-Coverage}
\end{center}

\section{Umgebung}
Die Code-Coverage wurde mit dem ant tool \textbf{cobertura} getestet. Hierfür wurden die Targets \texttt{instrument, coverage-check, coverage-report, alternate-coverage-report} zum ant-File hinzugefügt und das Target \texttt{test} entsprechend der Dokumentation von cobertura und einigen Forenseiten ergänzt. 


\section{ant all}
Lässt man das build-File ohne spezifizierte Target laufen, wird \texttt{all} ausgeführt. Dabei werden \textbf{drei} neue Ordner angelegt.
\begin{itemize}
\item[\textbf{build}] Die übersetzen Source-Files
\item[\textbf{instrumented}] Die zu traversierenden Class-Files für cubertura
\item[\textbf{report}] Cobertura erstellt hier für die durchlaufenen Tests xml- und html-Zusammenfassungen
\end{itemize}
\begin{center}
\includegraphics[scale=0.45]{ordner}
\end{center}
Ruft man nun \texttt{index.html} auf, öffnet sich eine Übersicht der getesten Files im Browser von der man sich durch die einzelnen Files manövrieren kann.
\begin{center}
\includegraphics[scale=0.24]{overview}
\end{center}
\newpage
\section{ArrayDequeTest}
\begin{center}
\includegraphics[scale=0.3]{arraydeque}

\end{center}
Der Code von ArrayDeque.java weißt initial eine line-coverage von 93\% auf. Sieht man sich den von cobertura markierten code an, werden drei Zeilen hervorgehoben:
\begin{center}
\begin{lstlisting}
return cap; //in capacity()
...
throw new DequeFull("Deque is full!"); //in addLast()
...
throw new DequeEmpty("Deque is empty!"); //in removeFirst()
\end{lstlisting} 
\end{center}
Für diese Fälle wurde die abstracte Klasse DequeTest um folgendeTests erweitert
\begin{lstlisting}
     @Test 
     public void TestH(){
        //capacity of emptyDeque liefert 10;
        assertEquals("capacity of konstructor and capacity() not equal", (long) 10, emptyDeque.capacity());
    }        
     @Test
     public void TestI() throws Exception{
     //addLast liefert fuer eine volle Deque eine Ausnahme.
      	 try{
       	    fullDeque.addLast(new Object());
       	    fail("addLast() did not throw exception");
      	  }catch (DequeFull e){}
   	 }
    @Test
   	 public void testJ() throws Exception{
     	//removeFirst liefert fuer eine leere Deque eine Ausnahme.
        try{
            emptyDeque.removeFirst();
            fail("removeFirst() did not throw exception");
        }catch (DequeEmpty e){}
     }
 

\end{lstlisting}
Nach dieser Maßnahme ist die Line-Coverage von ArrayDeque auf 100\% gestiegen.
\begin{center}
\includegraphics[scale=0.5]{ArrayDeque100}
\end{center}
\section{LinkedDequeTest}
Wie in Abbildung 2 zu sehen, hatte LinkedDeque eine Code-Coverage von 87\%. Da LinkedDequeTest und ArrayDequeTest von DeqeuTest erben, werden die neuen Tests auch für LinkedDeque ausgeführt und erhöhen dessen Line-Coverage auf 93\%.
\begin{center}
\includegraphics[scale=0.45]{linkeddeque}
\end{center}
Für die nicht abgedeckten Lines
\begin{lstlisting}
if (length == 0){ //in addFirst()
	first = new Item(e, null, null);
	last = first;
	length++;
\end{lstlisting}
wurde folgender Test geschrieben
\begin{lstlisting}
@Test 
public void testK() throws Exception{
    //Groesse von emptyDeque nach einfuegen von einem Element ist 1
	Object e = new Object();
    emptyDeque.addFirst(e);
    assertEquals("emptyDeque after insert length is not 1", (long) 1, emptyDeque.size());
}
\end{lstlisting}
Die Line Coverage von LinkedDeque beträgt nun auch 100\%
\begin{center}
\includegraphics[scale=0.5]{LinkedDeque100}
\end{center}
\end{document}
